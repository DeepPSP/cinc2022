% -*- Mode:TeX -*-

\documentclass{cinc-abstract}
\begin{document}

% The title is set in 14 point Helvetica bold
% \title{12-lead Electrocardiogram Arrhythmia Detection using Deep Neural Networks}
\title{Searching for Effective Neural Network Architecture for Heart Murmur Detection from Phonocardiogram}

% The rest of the title block is set in 12 point Helvetica
\author {Jingsu Kang, Hao Wen\\ % First name, initials and surnames, no ``and''
\ \\ % leave an empty line between authors and affiliation
Tianjin Medical University\\  % give affiliation of first author only
Tianjin, China} % city, [state or province,] country only

\maketitle

%%%%%%%%%%%%%%%%%%%%%%%%%%%%%%%%%%%%%%%%%%%%%%%%%%%%%%%%%%
% NOTE: The body of the abstract (exclusive of the title, authors, and authors’ affiliations) can be up to 300 words at most
%%%%%%%%%%%%%%%%%%%%%%%%%%%%%%%%%%%%%%%%%%%%%%%%%%%%%%%%%%


Aim: This work focuses on the problem of the detection of heart murmur from phonocardiogram recordings.

Methods: Considering the ...

Results: The best entry submission of our team ``Revenger'' received a score of 736 on the hidden validation set, ranked 32 out of 166 submissions.

Conclusion: to write

\end{document}
